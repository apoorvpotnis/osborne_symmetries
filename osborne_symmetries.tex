% !TEX program = lualatex

\documentclass[a4 paper, 12pt]{book}
\usepackage{geometry}

\usepackage[ngerman, italian, english]{babel}
\PassOptionsToPackage{math-style=ISO, bold-style=ISO, sans-style=italic, nabla=upright, partial=upright, warnings-off={mathtools-colon,mathtools-overbracket}}{unicode-math}
\usepackage{mathtools, xparse}
\usepackage{microtype}
\usepackage[newcmbb]{fontsetup}
\setmonofont[%
ItalicFont=NewCMMono10-Italic.otf,%
BoldFont=NewCMMono10-Bold.otf,%
BoldItalicFont=NewCMMono10-BoldOblique.otf,%
SlantedFont=NewCMMono10-Regular.otf,%
SlantedFeatures={FakeSlant=0.25},
BoldSlantedFont=NewCMMono10-Bold.otf,%
BoldSlantedFeatures={FakeSlant=0.25},
SmallCapsFeatures={Numbers=OldStyle}]{NewCMMono10-Regular.otf}
\setsansfont[%
SizeFeatures={%
	{Size= -8, Font=NewCMSans08-Regular.otf,%
		ItalicFont=NewCMSans08-Oblique.otf,%
		BoldFont=NewCMSans10-Bold.otf,%
		BoldItalicFont=NewCMSans10-BoldOblique.otf,%
		SmallCapsFeatures={Numbers=OldStyle},%
	},
	{Size= 8, Font=NewCMSans08-Regular.otf,%
		ItalicFont=NewCMSans08-Oblique.otf,%
		BoldFont=NewCMSans10-Bold.otf,%
		BoldItalicFont=NewCMSans10-BoldOblique.otf,%
		SmallCapsFeatures={Numbers=OldStyle},%
	},
	{Size= 9-, Font=NewCMSans10-Regular.otf,%
		ItalicFont=NewCMSans10-Oblique.otf,%
		BoldFont=NewCMSans10-Bold.otf,%
		BoldItalicFont=NewCMSans10-BoldOblique.otf,%
		SmallCapsFeatures={Numbers=OldStyle},%
	}},
ItalicFont=NewCMSans10-Oblique.otf,%
BoldFont=NewCMSans10-Bold.otf,%
BoldItalicFont=NewCMSans10-BoldOblique.otf,%
SmallCapsFeatures={Numbers=OldStyle},%
SlantedFont=NewCMSans10-Oblique.otf,%
BoldSlantedFont=NewCMSans10-BoldOblique.otf,%
]{NewCMSans10-Regular.otf}
%% math (upright) sans Greek
\renewcommand*{\msansAlpha}{\char"E000}
\renewcommand*{\msansBeta}{\char"E001}
\renewcommand*{\msansGamma}{\char"E002}
\renewcommand*{\msansDelta}{\char"E003}
\renewcommand*{\msansEpsilon}{\char"E004}
\renewcommand*{\msansZeta}{\char"E005}
\renewcommand*{\msansEta}{\char"E006}
\renewcommand*{\msansTheta}{\char"E007}
\renewcommand*{\msansIota}{\char"E008}
\renewcommand*{\msansKappa}{\char"E009}
\renewcommand*{\msansLambda}{\char"E00A}
\renewcommand*{\msansMu}{\char"E00B}
\renewcommand*{\msansNu}{\char"E00C}
\renewcommand*{\msansXi}{\char"E00D}
\renewcommand*{\msansOmicron}{\char"E00E}
\renewcommand*{\msansPi}{\char"E00F}
\renewcommand*{\msansRho}{\char"E010}
\renewcommand*{\msansSigma}{\char"E011}
\renewcommand*{\msansTau}{\char"E012}
\renewcommand*{\msansUpsilon}{\char"E013}
\renewcommand*{\msansPhi}{\char"E014}
\renewcommand*{\msansChi}{\char"E015}
\renewcommand*{\msansPsi}{\char"E016}
\renewcommand*{\msansOmega}{\char"E017}
\renewcommand*{\msansalpha}{\char"E018}
\renewcommand*{\msansbeta}{\char"E019}
\renewcommand*{\msansgamma}{\char"E01A}
\renewcommand*{\msansdelta}{\char"E01B}
\renewcommand*{\msansepsilon}{\char"E01C}
\renewcommand*{\msanszeta}{\char"E01D}
\renewcommand*{\msanseta}{\char"E01E}
\renewcommand*{\msanstheta}{\char"E01F}
\renewcommand*{\msansiota}{\char"E020}
\renewcommand*{\msanskappa}{\char"E021}
\renewcommand*{\msanslambda}{\char"E022}
\renewcommand*{\msansmu}{\char"E023}
\renewcommand*{\msansnu}{\char"E024}
\renewcommand*{\msansxi}{\char"E025}
\renewcommand*{\msansomicron}{\char"E026}
\renewcommand*{\msanspi}{\char"E027}
\renewcommand*{\msansrho}{\char"E028}
\renewcommand*{\msansvarsigma}{\char"E029}
\renewcommand*{\msanssigma}{\char"E02A}
\renewcommand*{\msanstau}{\char"E02B}
\renewcommand*{\msansupsilon}{\char"E02C}
\renewcommand*{\msansphi}{\char"E02D}
\renewcommand*{\msanschi}{\char"E02E}
\renewcommand*{\msanspsi}{\char"E02F}
\renewcommand*{\msansomega}{\char"E030}
\renewcommand*{\msansvarepsilon}{\char"E031}
% math italic sans Greek
\renewcommand*{\mitsansAlpha}{\char"E041}
\renewcommand*{\mitsansBeta}{\char"E042}
\renewcommand*{\mitsansGamma}{\char"E043}
\renewcommand*{\mitsansDelta}{\char"E044}
\renewcommand*{\mitsansEpsilon}{\char"E045}
\renewcommand*{\mitsansZeta}{\char"E046}
\renewcommand*{\mitsansEta}{\char"E047}
\renewcommand*{\mitsansTheta}{\char"E048}
\renewcommand*{\mitsansIota}{\char"E049}
\renewcommand*{\mitsansKappa}{\char"E04A}
\renewcommand*{\mitsansLambda}{\char"E04B}
\renewcommand*{\mitsansMu}{\char"E04C}
\renewcommand*{\mitsansNu}{\char"E04D}
\renewcommand*{\mitsansXi}{\char"E04E}
\renewcommand*{\mitsansOmicron}{\char"E04F}
\renewcommand*{\mitsansPi}{\char"E050}
\renewcommand*{\mitsansRho}{\char"E051}
\renewcommand*{\mitsansSigma}{\char"E052}
\renewcommand*{\mitsansTau}{\char"E053}
\renewcommand*{\mitsansUpsilon}{\char"E054}
\renewcommand*{\mitsansPhi}{\char"E055}
\renewcommand*{\mitsansChi}{\char"E056}
\renewcommand*{\mitsansPsi}{\char"E057}
\renewcommand*{\mitsansOmega}{\char"E058}
\renewcommand*{\mitsansalpha}{\char"E059}
\renewcommand*{\mitsansbeta}{\char"E05A}
\renewcommand*{\mitsansgamma}{\char"E05B}
\renewcommand*{\mitsansdelta}{\char"E05C}
\renewcommand*{\mitsansepsilon}{\char"E05D}
\renewcommand*{\mitsanszeta}{\char"E05E}
\renewcommand*{\mitsanseta}{\char"E05F}
\renewcommand*{\mitsanstheta}{\char"E060}
\renewcommand*{\mitsansiota}{\char"E061}
\renewcommand*{\mitsanskappa}{\char"E062}
\renewcommand*{\mitsanslambda}{\char"E063}
\renewcommand*{\mitsansmu}{\char"E064}
\renewcommand*{\mitsansnu}{\char"E065}
\renewcommand*{\mitsansxi}{\char"E066}
\renewcommand*{\mitsansomicron}{\char"E067}
\renewcommand*{\mitsanspi}{\char"E068}
\renewcommand*{\mitsansrho}{\char"E069}
\renewcommand*{\mitsansvarsigma}{\char"E06A}
\renewcommand*{\mitsanssigma}{\char"E06B}
\renewcommand*{\mitsanstau}{\char"E06C}
\renewcommand*{\mitsansupsilon}{\char"E06D}
\renewcommand*{\mitsansphi}{\char"E06E}
\renewcommand*{\mitsanschi}{\char"E06F}
\renewcommand*{\mitsanspsi}{\char"E070}
\renewcommand*{\mitsansomega}{\char"E071}
\renewcommand*{\mitsansvarepsilon}{\char"E072}
\setmathfont[range={\mathfrak,\mathbffrak}]{NewCMMath-Regular.otf}
\DeclareMathAlphabet\cmmathcal{OMS}{cmsy}{m}{n}
\renewcommand{\mathcal}{\cmmathcal}

\usepackage[style=british]{csquotes}

\usepackage{amsthm}
\theoremstyle{definition}
\newtheorem{theorem}{Theorem}[section]
\newtheorem{definition}[theorem]{Definition}
\newtheorem{example}[theorem]{Example}
\newtheorem{remark}[theorem]{Remark}

\newtheoremstyle{break}%
	{}{}%
	{}{}%
	{\bfseries}{}% % Note that final punctuation is omitted.
	{\newline}{}
\newtheorem{examples}[theorem]{Examples}

\begin{filecontents}{osborne_symmetries.bib}
	@book{Serre,
		author = {Serre, Jean-Pierre},
		isbn = {978-1-4684-9460-0},
		publisher = {Springer-Verlag, New York Inc.},
		title = {Linear Representations of Finite Groups},
		year = {1977},
		series = {Graduate Texts in Mathematics 42},
		addendum = {Translated from the French by Leonard L.\ Scott}
	}

	@book{Weinberg,
		author = {Weinberg, Steven},
		isbn = {978-0-521-55001-7},
		publisher = {Cambridge University Press, Cambridge},
		title = {The Quantum Theory of Fields: Foundations},
		volume = 1,
		year = {2005}
	}

	@book{Bowers,
		author = {Bowers, Philip},
		isbn = {978-1-108-42976-4},
		publisher = {Cambridge University Press, Cambridge},
		title = {Lectures on Quantum Mechanics},
		year = {2020}
	}

	@book{Moretti,
		author = {Moretti, Valter},
		isbn = {978-3-319-70705-1},
		publisher = {Springer International Publishing AG, Cham, Switzerland},
		series = {\selectlanguage{italian}La Matematica per il 3+2\selectlanguage{english}},
		title = {Spectral Theory and Quantum Mechanics: Mathematical Foundations of Quantum Theories, Symmetries and Introduction to the Algebraic Formulation},
		year = {2017},
		edition = 2,
		addendum = {Translated from Italian by Simon G.\ Choissi.}
	}

	@misc{Schuller_geometric_notes,
		author = {Schuller, Frederic and Rea, Simon and Dadhley, Richie},
		institution = {\selectlanguage{ngerman}Friedrich-Alexander-Universität Erlangen-Nürnberg, Institut für Theoretische Physik III\selectlanguage{english}},
		title = {Lectures on the Geometric Anatomy of Theoretical Physics},
		url = {https://drive.google.com/file/d/1nchF1fRGSY3R3rP1QmjUg7fe28tAS428/view},
		year = {2017},
		note = {Lecture notes in \texttt{.pdf} format. Lecturer: Prof.\@ Frederic Paul Schuller}
	}

	@misc{Schuller_geometric_videos,
		author = {Schuller, Frederic},
		title = {Lectures on the Geometric Anatomy of Theoretical Physics},
		url = {https://www.youtube.com/playlist?list=PLPH7f_7ZlzxTi6kS4vCmv4ZKm9u8g5yic},
		year = {2016},
		note = {Video lectures on YouTube.}
	}

	@misc{Osborne_quantum_essentials,
		author = {Osborne, Tobias},
		title = {Quantum mechanics essentials: Everything you need for quantum computation},
		url = {https://www.youtube.com/watch?v=28ABEInFxBQ},
		year = {2023},
		note = {Video lecture on YouTube.}
	}

	@misc{Osborne_wigner,
		author = {Osborne, Tobias},
		title = {Advanced quantum theory, Lecture 18},
		url = {https://www.youtube.com/watch?v=-Tb8B4_9a7g},
		year = {2016},
		note = {Video lecture on YouTube.}
	}

	@misc{Moretti_peter_weyl,
		author = {Valter Moretti (\url{https://physics.stackexchange.com/users/35354/valter-moretti})},
		howpublished = {Physics Stack Exchange (\url{https://physics.stackexchange.com})},
		note = {Version: 2024-05-17 21:03:48Z},
		title = {Answer to the question `Angular momentum Lie algebra for infinite-dimensional Hilbert spaces'},
		url = {https://physics.stackexchange.com/a/814882/81224}
	}

\end{filecontents}

\usepackage[sorting=none]{biblatex}
\addbibresource{osborne_symmetries.bib}
\usepackage{bibentry}

\usepackage{embedall}
\embedfile[desc = bibliography source file]{osborne_symmetries.bib}

\usepackage{imakeidx}
\makeindex[intoc]

\usepackage{hyperref}
\hypersetup{citecolor=red, pdfencoding=auto, psdextra, colorlinks=true, linkcolor=red, breaklinks=true, urlcolor=blue, pdftitle={Osborne's Lectures on Symmetries and Quantum Mechanics}, bookmarksopen=true, pdfauthor={Apoorv Potnis}, pdfsubject={Osborne's Lectures on Symmetries and Quantum Mechanics}, unicode=true, pdftoolbar=true, pdfmenubar=true, pdfkeywords={Tobias Osborne, Quantum Mechanics, Symmetries, Representation Theory, Linear Representations of Groups, Mathematical Physics, Lecture Notes}}
\usepackage{cleveref, xurl}
\usepackage[numbered]{bookmark}
\usepackage{booktabs}
\usepackage{array}
\usepackage{caption}
\usepackage{subcaption}
\usepackage{enumitem}

\usepackage{tikz}
\usetikzlibrary{datavisualization, datavisualization.polar,datavisualization.formats.functions, fpu, calc, arrows.meta, bending, positioning, 3d, quotes, angles, decorations.pathmorphing, backgrounds, fit, babel, hobby, decorations.markings}

\tikzset{%
	std line width/.style={
		line width = 0.7pt,
	},
	hollow circle/.style={
		draw, std line width, circle, inner sep=0pt, minimum size=3pt, outer sep=0pt
	}
}

\DeclarePairedDelimiterX\set[1]\lbrace\rbrace{\setaux#1}
\def\setaux#1|{#1\;\delimsize\vert\;}

\newcommand{\ltwo}{\mathup{L\kern-0.5pt^2}}
\newcommand{\position}{\mathup{Q}}
\newcommand{\momentum}{\mathup{P}}
\newcommand{\rthree}{\mathbb{R}^3}
\newcommand{\rr}{\mathbb{R}}
\newcommand{\cc}{\mathbb{C}}
\newcommand{\ii}{\mathbb{I}}
\newcommand{\nn}{\mathbb{N}_0}
\newcommand{\zz}{\mathbb{Z}}
\renewcommand{\ss}{\mathbb{S}}
\newcommand{\mm}{\mathbb{M}}
\newcommand{\dirac}{\mathup{\delta}}
\renewcommand*{\hbar}{\mathrm{^^^^0127}}
\renewcommand{\i}{\mathrm{i}}
\newcommand{\e}{\mathrm{e}}
\newcommand{\cinfinity}{\mathrm{C}^\infty}
\newcommand{\domain}{\mathcal{D}}
\newcommand{\identity}{\mathrm{id}}
\DeclarePairedDelimiter{\norm}{\lVert}{\rVert}
\DeclarePairedDelimiter{\abs}{\lvert}{\rvert}
\newcommand{\der}{\operatorname{d\!}{}}
\newcommand{\emm}{\mathcal{M}}
\newcommand{\oo}{\mathcal{O}}
\newcommand{\atlas}{\mathcal{A}}
\newcommand{\powerset}{\mathcal{P}}
\newcommand{\ball}{\mathrm{B}}
\DeclareMathOperator{\GL}{GL}
\newcommand{\basis}{\mathcal{B}}
\newcommand{\hilbert}{\mathcal{H}}
\newcommand{\unitary}{\mathcal{U}}
\newcommand{\ortho}{\mathrm{O}}
\newcommand{\lorentz}{\ortho{(3,1)}}
\newcommand{\man}{\mathcal{M}}
\newcommand{\euc}{\mathrm{E}}
\DeclareMathOperator{\isom}{Isom}
\newcommand{\mink}{\mm_{3,1}}
\newcommand{\poin}{\mathrm{IO}(3,1)}

\DeclarePairedDelimiterX{\bra}[1]{\langle}{\rvert}{#1}
\DeclarePairedDelimiterX{\ket}[1]{\lvert}{\rangle}{#1}
\DeclarePairedDelimiterX\braket[2]{\langle}{\rangle}{#1\delimsize\vert\mathopen{}#2}

\title{\textsf{\textbf{Osborne's Lectures on Symmetries and Quantum Mechanics}}}
\author{}
\date{\textsf{\today}}

\begin{document}
	\hypertarget{TitlePage}{}
	\bookmark[dest=TitlePage]{Title Page}
	\maketitle

	\chapter*{Preface}
	\hypertarget{Preface}{}
	\bookmark[dest=Preface]{Preface}
	These are lecture notes by Apoorv Potnis of the lecture series `Symmetries and Quantum Mechanics', given by \textbf{Prof.\ Tobias J.\ Osborne} in 2023 at the \selectlanguage{ngerman}Leibniz Universität Hannover\selectlanguage{english}. Prof.\ Osborne discusses the basics of the representation theory of groups in the context of quantum mechanics in this short lecture series. The video lecture series is available at \url{https://youtube.com/playlist?list=PLDfPUNusx1ErdQhrdAzincNJKgTQahsX_&feature=shared}.

	The source code, updates and corrections to this document can be found on this GitHub repository: \url{https://github.com/apoorvpotnis/osborne_symmetries}. The source code is embedded in this \textsc{pdf}. Comments and corrections can be mailed at \href{mailto:apoorvpotnis@gmail.com}{\texttt{apoorvpotnis@gmail.com}}.
	\clearpage

	\hypertarget{Contents}{}
	\bookmark[dest=Contents]{Contents}
	\tableofcontents

	\chapter{Basics, Wigner's theorem and linear representation of groups}
	\chaptermark{Basics}

	\section{Prerequisites and references}

	We assume that the reader has a working knowledge of linear algebra and is familiar with basic ideas of quantum mechanics. Mainly, one needs the knowledge of the postulates of quantum mechanics, which can be learnt from Prof.\ Osborne's video lecture on YouTube titled `Quantum mechanics essentials: Everything you need for quantum computation'~\cite{Osborne_quantum_essentials}.

	The main reference book this lecture course series is based on is the following famous book of Serre.\begin{quote}\fullcite{Serre}.\end{quote}The first part of this book is based on lectures given by the eminent mathematician Jean-Pierre Serre to a group of quantum chemists. One may look at the first quantum field theory volume of Weinberg, but it is too advanced for our present purposes and one would have great difficulties reading it.\begin{quote}\fullcite{Weinberg}.\end{quote}

	\section{Postulates of quantum mechanics}

	We briefly state the postulates of quantum mechanics.
	\begin{enumerate}
	    \item A Hilbert space corresponds to every quantum mechanical system.
		\item The states of quantum mechanical systems are represented by density matrices.
		\item The measurements or detectors are represented by positive operators.
		\item B\"orn rule.
		\item Schr\"odinger's equation.
		\item Tensor product for composite systems.
	\end{enumerate}
	We shall focus on the fifth postulate, namely the Schr\"odinger equation. We argue that we don't actually need it and it can be extracted from the other postulates.

	\section{Symmetries}

	A \index{Symmetry!of a quantum system}symmetry on a quantum system is a physical operation that can be performed or can occur. The most general operation in quantum mechanics that can occur on a a system is represented by a completely positive map.\footnote{But then how are symmetries different from a general completely positive map?} Symmetries thus need to be completely positive maps as well. In fact, we shall argue that symmetries form a small subset of completely positive maps.
	Take a system, and wait for some time \(t \in \rr\). It is possible for a system to not change in that time. Thus, `waiting for time \(t\)' is a symmetry. For every time \(t \in \rr\), there exists a possible symmetry operation, namely, waiting for time \(t\). Thus, we get a set of symmetries labelled by \(t\). What we are actually are interested in this course are sets of symmetries. We call the set of all labels as \(G\). There are some properties for symmetries which we desire, as follows.
	\begin{enumerate}
	    \item The symmetry of doing nothing must always be present in the set of symmetries.
		\item If we have two symmetries, then we must be able to do them one after the other, and the resulting operation must be a symmetry as well.
		\item If we have a symmetry operation, we must be able to `reverse' the operation to get the system back to its initial state.
	\end{enumerate}
	It should be noted that these symmetry operations are the in principle the `possible' operations on a system, operations that we can think of, at the very least. It may be extremely difficult, or impossible even, to actually perform these operations on a physical system.

	In order to capture the three requirements that the set of symmetries must possess, we formalise the notion of an algebraic group, which the reader will have no doubt encountered before.
	\begin{definition}[\index{Group}Group]
	    A group is a set \(G\), together with a law of composition, i.e.\ a composition map
		\begin{align*}
		    \circ &\colon G \times G \rightarrow G,\\
			\circ &\colon (x, y) \mapsto x \circ y,
		\end{align*}
		such that
		\begin{enumerate}
		    \item \((x\circ y) \circ z = x\circ (y \circ z)\), i.e.\ the product is associative,
			\item \(G\) contains a unit element \(1\) such that \(x \circ 1 = 1 \circ x = x\), for all \(x \in G\), and,
			\item for all \(x\in G\), there exists an inverse \(y \in G\) such that \(x \circ y = y \circ x = 1\).
		\end{enumerate}
	\end{definition}
	\noindent We would often drop the \(\circ\) and denote \(x\circ y\) as simply \(xy\).
	\begin{examples}
		\leavevmode \vspace{-\baselineskip}\vspace{12pt}
		\begin{enumerate}[label={(\alph*)},ref={\alph*}]
			\item \(\zz / 2\zz\) is a group consisting of two elements; \(\{0, 1\}\) with addition operation \(\oplus\) defined as the usual addition modulo \(2\). It is the group of reflections, and physically captures the symmetry of charge conjugation, parity, etc.
			The group multiplication table is as follows.
			\begin{align*}
				\setlength{\extrarowheight}{3pt}% local setting
				\begin{array}{l|*{5}{l}}
						& 0   & 1\\
					\hline
					0   & 0   & 1\\
					1   & 1   & 0\\
				\end{array}
			\end{align*}
			\item \(\zz\) is a group with the addition operation. Similarly, \(\zz \times \zz\) is a group too. This can represent the symmetries of a lattice of atoms.
			\item \(\rr\) is a group with the addition operation. This captures the time and space translation symmetries.
			\item The symmetries of the square, the dihedral group \(\mathrm{D}_4\), of order \(8\), forms a group as well. The order of a finite group is the number of elements it contains. The group consists of rotations and flips of a square. This group is non-Abelian. A group is said to be Abelian if for all \(x, y \in G\), we have that \(xy = yx\).
			\item Let \(V\) be a finite-dimensional complex vector space. We denote by \(\GL(V)\) the group of isomorphisms of \(V\) with itself, i.e.\ the group of all invertible linear maps \(a \colon V \rightarrow V\). Such maps may be identified with matrices, with order \(n \times n\), when \(\dim (V) = n < \infty\). Let \(\basis = \{\ket{e_i} \mid j = 1, \ldots, n\}\) be a basis. Then we have that
			\begin{align*}
				a\ket{e_j} = \sum_{k=1}^n a_{kj}\ket{e_k}.
			\end{align*}
			\item Let \(\hilbert\) be a Hilbert space. Then the set \(\unitary(\hilbert)\) of unitary operators acting on \(\hilbert\) is a group. It is an extremely important group in quantum mechanics as it effectively forms the set of symmetries of a quantum system.\footnote{But Prof.\ Osborne then immediately remarks that there exist symmetries which are not unitaries. I don't understand how this is not a contradiction.}
		\end{enumerate}
	\end{examples}

	All the known symmetries in physics are captured by groups, thus \textbf{symmetries of physical systems will be labelled by groups.} It should be noted that there are some speculations that one might need some different structure such as a fusion category, but this is advanced stuff that we won't discuss in this course. It is not yet clear whether fusion categories have physical relevance.

	We already have an intuitive idea of what it means for two groups `to be basically the same'. We now formalise that.
	\begin{definition}[\index{Group!homomorphism}Group homomorphism]
	    Let \(G\) and \(H\) be two groups. A group homomorphism of \(G\) into \(H\) is a map \(f \colon G \rightarrow H\) such that \(f(gh) = f(g) f(h)\), for all \(g, h \in G\), and \(f(1_G) = 1_H\).
	\end{definition}

	\begin{example}
		Let \(f \colon \zz/2\zz \rightarrow \{\ii, c\}\) by \(f(0) = \ii\) and \(f(1) = c\), where \(c\) denotes charge conjugation. Then \(f\) is a group homomorphism.
	\end{example}

	We now contrast the notion of a symmetry with something known as a \textit{symmetry transformation}.
	\begin{definition}[\index{Symmetry!transformation}Symmetry transformation]
		A symmetry transformation \(T \colon \hilbert \rightarrow \hilbert\) is an invertible transformation of rays in a Hilbert space \(\hilbert\) which preserves the transition probabilities, i.e.\ for all \(\ket{\psi} \in [\ket{\psi}]\) and \(\ket{\phi} \in [\ket{\phi}]\), with \(\ket{\psi'} \in [\ket{\psi}]\) and \(\ket{\phi'} \in [\ket{\phi}]\),
		\begin{align*}
			{\abs{\braket{\psi'}{\phi'}}}^2 = {\abs{\braket{\psi}{\phi}}}^2.
		\end{align*}
	\end{definition}
	\noindent Note that we haven't explicitly demanded that \(T\) be linear, however the famous Wigner's theorem forces the symmetry transformations to be linear.

	\section{Linear representations}

	\begin{definition}[\index{Representation!Linear}\index{Representation!Unitary}Linear and unitary representations]
		Suppose \(G\) is a finite group with the identity \(1\), and let \(V\) be a complex vector space. A linear representation of \(G\) in \(V\) is a group homomorphism \(\rho\) from \(G\) into \(\GL(V)\), i.e.\ \(\rho(1) = \ii_n\) and \(\rho(st) = \rho(s) \rho(t)\) for all \(s, t \in G\). A linear representation is said to be unitary if \(\rho(G) \subset \unitary(\hilbert) \subset \GL(V)\).
	\end{definition}
	\begin{theorem}[Wigner]
		Any symmetry transformation \(T\) has a representation on \(\hilbert\) as an operator which is either unitary or anti-unitary.\footnote{An anti-unitary transformation \(U\) is a transformation for which \(U(a\ket{\psi}) = a^*(U\ket{\psi})\).}
	\end{theorem}
	The proof of Wigner's theorem is not discussed in this course. Prof.\ Osborne discusses the proof in the eighteenth lecture of his `Advanced Quantum Mechanics' course, available on YouTube~\cite{Osborne_wigner}.

	We now discuss an important non-example. Let \(\hilbert = \cc^4\) and \(G = \lorentz\) be the group of Lorentz transformations. There does not exist a \(\rho \colon \lorentz \rightarrow \unitary(\cc^4)\) that is non-trivial. In fact, there is no non-trivial unitary representation of the Lorentz group on any finite-dimensional Hilbert space. Thus, when one tries to combine special relativity and quantum mechanics, one is inevitably drawn to infinite-dimensional Hilbert spaces.

	\chapter{Linear representations}

	\section{Goals for the course}

	For the remaining part of the course, we shall try to construct `minimal' quantum mechanical systems which posses the particular symmetry we are interested in. Once we have that, we can build other, more complicated systems from them. Finally, we can attempt to classify the systems according to their symmetry groups. A long term goal in theoretical physics is to find out that given a space-time metric manifold \(\man\), what quantum systems `correspond to' \(\man\)? This is a very hard question which is researched still today. By `correspond to' we mean that for every `achievable' operation on \(\man\), there is a corresponding quantum mechanical operation implementing that operation. It turns out that the correct choice for `achievable' operations permitted on the manifold are the ones which preserve distances on the manifold. Thus, these operations form the isometry group of \(\man\), \(\isom(\man)\).
	\begin{examples}
		\leavevmode \vspace{-\baselineskip}\vspace{12pt}
		\begin{enumerate}
		    \item Let \(\man = \rr^3\). Then \(\isom(\rr^3) = \euc(3)\), the Euclidean group consisting of all the translations, rotations and reflections of \(\rr^3\). One can identify \(\rr^3\) with \(\euc(3)\): if we have a single point and apply all the operations available to us from \(\euc(3)\), we get the entire space.
			\item Let us take \(\man\) to be the Minkowski space \(\mink\), which we identify with \(\rr^{3+1}\), in standard co-ordinates. Recall that the metric on \(\mink\) is given by the space-time interval. The isometries of \(\mink\) form the Poincaré group \(\poin\).
			\item If we take \(\ss^2\) to be the manifold, with distances given by great circles, we have \(\ortho(3)\) as the isometries.
		\end{enumerate}
	\end{examples}

	\nocite{*}
	\printbibliography[heading=bibintoc]
	\printindex
\end{document}

