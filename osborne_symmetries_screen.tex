% !TEX program = lualatex

\documentclass[12pt, oneside]{book}
\usepackage[paperwidth=15cm, paperheight=16.2cm, left=1.3cm, right=1.3cm, bottom=1.5cm, top=2cm]{geometry}

\usepackage[ngerman, italian, english]{babel}
\PassOptionsToPackage{math-style=ISO, bold-style=ISO, sans-style=italic, nabla=upright, partial=upright, warnings-off={mathtools-colon,mathtools-overbracket}}{unicode-math}
\usepackage{mathtools}
\usepackage{microtype}
\emergencystretch=1em
\usepackage[newcmbb]{fontsetup}
\usepackage{concmath-otf}

\usepackage[style=british]{csquotes}

\usepackage{amsthm}
\theoremstyle{definition}
\newtheorem{thm}{Theorem}
\newtheorem{defn}{Definition}
\newtheorem{exmp}{Example}
\newtheorem{remark}{Remark}

\begin{filecontents}{osborne_symmetries.bib}
	@book{Serre,
		author = {Serre, Jean-Pierre},
		isbn = {978-1-4684-9460-0},
		publisher = {Springer-Verlag, New York Inc.},
		title = {Linear Representations of Finite Groups},
		year = {1977},
		series = {Graduate Texts in Mathematics 42},
		addendum = {Translated from the French by Leonard L.\ Scott}
	}

	@book{Weinberg,
		author = {Weinberg, Steven},
		isbn = {978-0-521-55001-7},
		publisher = {Cambridge University Press, Cambridge},
		title = {The Quantum Theory of Fields: Foundations},
		volume = 1,
		year = {2005}
	}

	@book{Bowers,
		author = {Bowers, Philip},
		isbn = {978-1-108-42976-4},
		publisher = {Cambridge University Press, Cambridge},
		title = {Lectures on Quantum Mechanics},
		year = {2020}
	}

	@book{Moretti,
		author = {Moretti, Valter},
		isbn = {978-3-319-70705-1},
		publisher = {Springer International Publishing AG, Cham, Switzerland},
		series = {\selectlanguage{italian}La Matematica per il 3+2\selectlanguage{english}},
		title = {Spectral Theory and Quantum Mechanics: Mathematical Foundations of Quantum Theories, Symmetries and Introduction to the Algebraic Formulation},
		year = {2017},
		edition = 2,
		addendum = {Translated from Italian by Simon G.\ Choissi.}
	}

	@misc{Schuller_geometric_notes,
		author = {Schuller, Frederic and Rea, Simon and Dadhley, Richie},
		institution = {\selectlanguage{ngerman}Friedrich-Alexander-Universität Erlangen-Nürnberg, Institut für Theoretische Physik III\selectlanguage{english}},
		title = {Lectures on the Geometric Anatomy of Theoretical Physics},
		url = {https://drive.google.com/file/d/1nchF1fRGSY3R3rP1QmjUg7fe28tAS428/view},
		year = {2017},
		note = {Lecture notes in \texttt{.pdf} format. Lecturer: Prof.\@ Frederic Paul Schuller}
	}

	@misc{Schuller_geometric_videos,
		author = {Schuller, Frederic},
		title = {Lectures on the Geometric Anatomy of Theoretical Physics},
		url = {https://www.youtube.com/playlist?list=PLPH7f_7ZlzxTi6kS4vCmv4ZKm9u8g5yic},
		year = {2016},
		note = {Video lectures on YouTube.}
	}

	@misc{Osborne_quantum_essentials,
		author = {Osborne, Tobias},
		title = {Quantum mechanics essentials: Everything you need for quantum computation},
		url = {https://www.youtube.com/watch?v=28ABEInFxBQ},
		year = {2023},
		note = {Video lectures on YouTube.}
	}

	@misc{morettipeterweyl,
		author = {Valter Moretti (\url{https://physics.stackexchange.com/users/35354/valter-moretti})},
		howpublished = {Physics Stack Exchange (\url{https://physics.stackexchange.com})},
		note = {(version: 2024-05-17 21:03:48Z)},
		title = {Answer to the question `Angular momentum Lie algebra for infinite-dimensional Hilbert spaces'},
		url = {https://physics.stackexchange.com/a/814882/81224}
	}

\end{filecontents}

\usepackage[sorting=none]{biblatex}
\addbibresource{osborne_symmetries.bib}
\usepackage{bibentry}

\usepackage{embedall}
\embedfile[desc = bibliography source file]{osborne_symmetries.bib}

\usepackage{imakeidx}
\makeindex[intoc]

\usepackage{hyperref}
\hypersetup{pdfpagelayout=TwoPageLeft, citecolor=red, pdfencoding=auto, psdextra, colorlinks=true, linkcolor=red, breaklinks=true, urlcolor=blue, pdftitle={Osborne's Lectures on Symmetries and Quantum Mechanics}, pdfauthor={Apoorv Potnis}, pdfsubject={Osborne's Lectures on Symmetries and Quantum Mechanics}, unicode=true, pdftoolbar=false, pdfstartview={FitV}, pdfview=FitV, pdfkeywords={Tobias Osborne, Quantum Mechanics, Symmetries, Representation Theory, Linear Representations of Groups, Mathematical Physics, Lecture Notes}, pdfpagemode=FullScreen, pdflang={English}}
\usepackage{cleveref, xurl}
\usepackage[numbered]{bookmark}

\usepackage{booktabs}
\usepackage{array}
\usepackage{caption}
\usepackage{subcaption}

\usepackage{tikz}
\usetikzlibrary{datavisualization, datavisualization.polar,datavisualization.formats.functions, fpu, calc, arrows.meta, bending, positioning, 3d, quotes, angles, decorations.pathmorphing, backgrounds, fit, babel, hobby, decorations.markings}

\tikzset{%
	std line width/.style={
		line width = 0.7pt,
	},
	hollow circle/.style={
		draw, std line width, circle, inner sep=0pt, minimum size=3pt, outer sep=0pt
	}
}

\DeclarePairedDelimiterX\set[1]\lbrace\rbrace{\setaux#1}
\def\setaux#1|{#1\;\delimsize\vert\;}

\newcommand{\ltwo}{\mathup{L\kern-0.5pt^2}}
\newcommand{\position}{\mathup{Q}}
\newcommand{\momentum}{\mathup{P}}
\newcommand{\rthree}{\mathbb{R}^3}
\newcommand{\rr}{\mathbb{R}}
\newcommand{\cc}{\mathbb{C}}
\newcommand{\nn}{\mathbb{N}_0}
\newcommand{\dirac}{\mathup{\delta}}
\renewcommand*{\hbar}{\mathrm{^^^^0127}}
\renewcommand{\i}{\mathrm{i}}
\newcommand{\e}{\mathrm{e}}
\newcommand{\cinfinity}{\mathrm{C}^\infty}
\newcommand{\domain}{\mathcal{D}}
\newcommand{\identity}{\mathrm{id}}
\DeclarePairedDelimiter{\norm}{\lVert}{\rVert}
\DeclarePairedDelimiter{\abs}{\lvert}{\rvert}
\newcommand{\der}{\operatorname{d\!}{}}
\newcommand{\emm}{\mathcal{M}}
\newcommand{\oo}{\mathcal{O}}
\newcommand{\atlas}{\mathcal{A}}
\newcommand{\powerset}{\mathcal{P}}
\newcommand{\ball}{\mathrm{B}}

\title{Osborne's Lectures on Symmetries and Quantum Mechanics}
\author{}
\date{\today}

\begin{document}
	\hypertarget{TitlePage}{}
	\bookmark[dest=TitlePage]{Title Page}
	\maketitle

	\vspace{-2em}
	\chapter*{\vspace{-2em}Note}
	\hypertarget{Note}{}
	\bookmark[dest=Note]{Note}
	This document is best viewed in a full-screen mode with two pages side-by-side, on a screen of aspect ratio 16:9.

	\vspace{-2em}
	\chapter*{\vspace{-2em}Preface}
	\hypertarget{Preface}{}
	\bookmark[dest=Preface]{Preface}
	These are lecture notes by Apoorv Potnis of the lecture series `Symmetries and Quantum Mechanics', given by \textbf{Prof.\ Tobias J.\ Osborne} in 2023 at the \selectlanguage{ngerman}Leibniz Universität Hannover\selectlanguage{english}. Prof.\ Osborne discusses the basics of the representation theory of groups in the context of quantum mechanics in this short lecture series. The video lecture series is available at \url{https://youtube.com/playlist?list=PLDfPUNusx1ErdQhrdAzincNJKgTQahsX_&feature=shared}.

	The source code, updates and corrections to this document can be found on this GitHub repository: \url{https://github.com/apoorvpotnis/osborne_symmetries}. The source code is embedded in this \textsc{pdf}. Comments and corrections can be mailed at \href{mailto:apoorvpotnis@gmail.com}{\texttt{apoorvpotnis@gmail.com}}.
	\clearpage

	\hypertarget{Contents}{}
	\bookmark[dest=Contents]{Contents}
	\tableofcontents

	\chapter{Basics, Wigner's theorem and linear representation of groups}
	\chaptermark{Basics}

	\section{Prerequisites and references}

	We assume that the reader has a working knowledge of linear algebra and is familiar with basic ideas of quantum mechanics. Mainly, one needs the knowledge of the postulates of quantum mechanics, which can be learnt from Prof.\ Osborne's video lecture on YouTube titled `Quantum mechanics essentials: Everything you need for quantum computation'~\cite{Osborne_quantum_essentials}.

	The main reference book this lecture course series is based on is the famous book of Serre.\begin{quote}\fullcite{Serre}.\end{quote}The first part of this book is based on lectures given by the eminent mathematician Jean-Pierre Serre to a group of quantum chemists. One may look at the first quantum field theory volume of Weinberg, but this is too advanced for present purposes and one would have great difficulties reading it.\begin{quote}\fullcite{Weinberg}.\end{quote}

	\section{Postulates of quantum mechanics}

	We briefly state the postulates of quantum mechanics.
	\begin{enumerate}
	    \item A Hilbert space corresponds to every quantum mechanical system.
		\item The states of quantum mechanical systems are represented by density matrices.
		\item The measurements or detectors are represented by positive operators.
		\item B\"orn rule.
		\item Schrodinger's equation.
		\item Tensor product for composite systems.
	\end{enumerate}
	We shall focus on the fifth postulate, namely the Schr\"odinger equation. We argue that we don't actually need it. It can be derived from deeper principles.

	\section{Symmetries}

	A symmetry on a quantum system is a physical operation that can be performed or can occur.

	\nocite{*}
	\printbibliography[heading=bibintoc]
	\printindex
\end{document}

